% amdg
\documentclass{article}

% use t hese as templates:
% https://aaai.org/ocs/index.php/AAAI/AAAI18/paper/view/16970/16653
% https://aaai.org/ocs/index.php/AAAI/AAAI18/paper/view/16932/16630
% https://aaai.org/ocs/index.php/AAAI/AAAI18/paper/view/17022/16605 

\title{Curriculum Learning}
\author{John Lalor} 

\begin{document}
\maketitle 
\begin{abstract}
CL has been shown to be effective.
Many measures of ``difficulty'' are based on heuristics and not on true difficulty.
we want to change that by leveraging methods from psychometrics.
Experiments on vision and language data show that using learned difficulty and ability to set up a curriculum is effective.
\end{abstract}

\section{Introduction}
- go over CL
- why it works
- why there are currently issues with it (heuristics)
- bottlenecks to some of the existing approaches

\section{Methods}

- brief review of IRT (focus on rasch model because that is what we're using)

- talk about how to fit the model using variational inference

- selecting data as the model is reayd for it (based on theta)

\section{Data and experiments} 

- vision (mnist and cifar)

- language (SNLI and SSTB) 

- experiments: ordered vs simple vs irt, balanced vs not balanced 

\section{Results} 

- plots and tables 

- really show that using our method is efficient and practical 

\section{Related work}

Lots to do here. need a thorough lit review as part of this paper 

\section{Conclusion} 


\end{document}