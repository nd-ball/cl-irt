% amdg
\documentclass{article}

% use t hese as templates:
% https://aaai.org/ocs/index.php/AAAI/AAAI18/paper/view/16970/16653
% https://aaai.org/ocs/index.php/AAAI/AAAI18/paper/view/16932/16630
% https://aaai.org/ocs/index.php/AAAI/AAAI18/paper/view/17022/16605 

\title{Curriculum Learning}
\author{John Lalor} 

\begin{document}
\maketitle 
\begin{abstract}
CL has been shown to be effective.
Many measures of ``difficulty'' are based on heuristics and not on true difficulty.
we want to change that by leveraging methods from psychometrics.
Experiments on vision and language data show that using learned difficulty and ability to set up a curriculum is effective.
Curriculum that is dynamically determined by the current competency of the model as estimated by Item Response Theory. 
\end{abstract}

\section{Introduction}
- go over CL
- why it works
- why there are currently issues with it (heuristics)
- bottlenecks to some of the existing approaches

Curriculum learning is a well-studied area of artificial intelligence, with results demonstrating that building a curriculum of training data where easy examples are shown first can speed up learning and improve generalization, as has also been shown in humans. 
The basic premise is that machine learning (ML) models are trained according to a ``curriculum'' that sorts training examples according to difficulty.
At first, the model is trained with only the easiest examples, and more difficult examples are gradually added according to some schedule.
A major gain for CL methods is that model convergence can often be quicker when the model is trained according to some curriculum (CITE a bunch of examples here).
Model efficiency has always been a major aim of the research community (cite old papers), but recently with the size of models continuing to grow and better understanding of the impact of model training on the environment there is a renewed need for efficiency (CITE EMMA).

A major drawback of existing CL techniques is that they rely on heuristics to measure the difficulty of data, and either ignore the competency of the model at its present state or rely on heuristics there as well.
For example, often for natural language processing (NLP) tasks, sentence length is considered a proxy for difficulty (cite naacl19 and original paper).
Similarly, the original CL paper used the number of objects in an image as a proxy for difficulty in an image recognition task (CCITE Bengio).
Competency was recently introduced as a mechanism to dictate when new examples should be added to the training data (CITE NAACL 19), however in that work the competency schedule was ad hoc, and did not actually look at the competency of the model but assumed a schedule according to learning heuristics (CITE NAACL). 
It would be better if model competency was actually measured, so that the training data could be appropriately matched with the model at a given point in the training.

Recent work has shown that it is possible to estimate both the difficulty of examples and the ability of deep learning models as latent variables based on model performance (cite emnlp19).
Item Response Theory (IRT) is a well-studied methodology in the psychometric literature for test set construction and subject evaluation.
A typical IRT model will estimate latent parameters such as difficulty for examples under consideration for inclusion in a test set.
This is done by administering a test to a large number of human subjects, collecting and grading their responses as correct or incorrect, and using the student-response data matrix to estimate the latent traits of the data.
Once learned, these latent parameters can be used to estimate latent ability parameters of future test-takers, based on their graded responses to the examples.
IRT has not seen wide adoption in the ML community, primarily due to the fact that fitting IRT models typically requires a large amount of human annotated data for each example.
However, recent work has shown that it is possible to fit IRT models using machine-generated data instead of human-generated data as the input to the IRT models (CITE US).
Because one can learn example difficulty and subject ability together, IRT is an interesting framework to consider for the problem of CL. 

In this work we propose Dynamic Curriculum Learning with IRT (DCL-IRT), a novel framework that uses the estimated ability of a model at a specific point in the training process to identify appropriate training data.
At each training epoch, the latent ability of the model is estimated.
Based on this estimate, only training data that the model has a reasonable chance of labeling correctly is included in training.
As the model improves, the estimated ability will improve, and more training data will be added.

Our contributions are as follows: (i) we propose a novel CL framework, DCL-IRT, which automatically selects training data based on the estimated ability of the model, (b) we show that model training using DCL-IRT leads to faster convergence than traditional training and better performance than baseline CL methods, (c) we provide an analysis of the DCL-IRT training regime to show why certain training examples hurt instead of help generalization.
This is the first work to learn a model competency during training that is directly comparable to the difficulty of the training data pool.

All code used for this work will be released upon publication, and are included as supplemental material.

\section{Methods}

- brief review of IRT (focus on rasch model because that is what we're using)

- talk about how to fit the model using variational inference

- selecting data as the model is ready for it (based on theta)
\subsection{Curriculum Learning}
In a traditional CL framework, training data examples are ordered according to some notion of difficulty, and the training set shown to the learner is augmented at a set pace with more and more difficult examples.

Typically, the model's current performance is not taken into account (CONFIRM THAT THIS IS TRUE FOR EVERYTHING BUT NAACL PAPER). Recent work has incorporated a notion of ``competency,'' but in actual fact just used a slightly more complex inclusion rate calculation that does not actually consider the competency of the model but simulates it.

\subsection{Learning Latent Parameters with IRT}
Learning the latent difficulty parameters of training examples can be done off-line using existing techniques (CITE EMNLP PAPER).
Traditional IRT model fitting relies on marginal maximum likelihood (MML) estimation, where the latent ability parameters ($\theta$) are assumed to be random effects and integrated out.
Latent item parameters are estimated using an expectation maximization (EM) method (CITE BOCK and AIKEN).
These methods assume human subjects, and were typically used for relatively small data sets of no more than a hundred items or so.
A major bottleneck of using IRT methods on machine learning data sets is the fact that each subject would have to label each (or most of) the examples in order to have enough response patterns to estimate the latent parameters.
Asking humans to annotate tens or hundreds of thousands of examples can't be done, but recent work has shown that the human subjects can be replaced with an ensemble of machine learning models, and the response patterns from this ``artificial crowd'' can be used to estimate item parameters.
Such models can be fit using variational inference methods, where the latent parameters are estimated via variational distributions (CITE Minka and Us). 

Estimating the ability of a model at a point in time is done with a ``scoring'' function. 
Since the difficulties of the examples are known, we can maximize the likelihood of the data given the response patterns to obtain the ability estimate.
All that is required is a single forward pass of the model on the data, as is typically done with a test set. 

\subsection{Dynamic Curriculum Learning with IRT}

We propose DCL-IRT, where the training examples are selected dynamically at each training epoch based on the estimated ability of the model at that point.

The first step of DCL-IRT is to estimate the ability of the model using the scoring function (Sec. above). To do this we use the full training set, but crucially, only to get response data, not to update parameters. 

Once ability is estimated, data selection is done by comparing esimated ability to the examples' difficulty parameters.

With DCL-IRT, the training data size does not have to be monotonically increasing. If a model's performance suffers as a result of adding data too quickly, then this will be reflected in lower ability estimates, which leads to less data selected in the next epoch. 

Algorithm XX shows all of the steps for DCL-IRT. Code implementing DCL-IRT is included as supplemental material and will be released upon publication. 


\section{Data and experiments} 

- vision (mnist and cifar)

- language (SNLI and SSTB) 

- experiments: ordered vs simple vs irt, balanced vs not balanced 

\section{Results} 

- plots and tables 

- really show that using our method is efficient and practical 

\section{Related work}

Lots to do here. need a thorough lit review as part of this paper 

\section{Conclusion} 


\end{document}